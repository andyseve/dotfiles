% Common Micros
\newcommand\ts{\textstyle}
\newcommand\ds{\displaystyle} 
\newcommand\dsum{\displaystyle\sum}
\newcommand\ubar[1]{\text{\underbar{$#1$}}}
\newcommand\nref[1]{\ref{#1}~\nameref{#1}}
\newcommand\rem{\textbf{\underline{Remark}}}
\newcommand\ques{\textbf{\underline{Question}}}
\newcommand\textbu[1]{\text{\textbf{\underline{#1}}}} % Combines bold and underline
\newcommand\tageqn{\addtocounter{equation}{1}\tag{\theequation}} % Equation numbering
\newcommand\labeleqn[1]{\addtocounter{equation}{1}\tag{\theequation}\label{#1}}
\newcommand\numbereqn{\addtocounter{equation}{1}\tag{\theequation}}

% Better symbols and renamings
\newcommand\Oh{\ensuremath{\mathcal{O}}}
\newcommand\oh{\scalebox{0.7}{$\mathcal{O}$}}
\newcommand\OhStar{\ensuremath{\Oh^{\star}}}
\newcommand\xor{\oplus}
\newcommand\tensor{\otimes} % I don't like to remember things.
\newcommand\transpose{\top} % Much better
\newcommand\directsum{\oplus}
\newcommand\iso{\cong}
\newcommand\del{\partial}
\newcommand\grad{\nabla}
\newcommand\conj[1]{\overline{#1}}
\newcommand\one{\mathbbm{1}}
\renewcommand\emptyset{\varnothing} % Slightly better symbol
\renewcommand\iff{\ensuremath{\Leftrightarrow}}

\newcommand{\pl}[0]{\partial}
\newcommand{\dd}[2]{\frac{d #1}{d #2}}
\newcommand{\ddd}[1]{\frac{d}{d #1}}
\newcommand{\pd}[2]{\frac{\partial #1}{\partial #2}}
\newcommand{\pdd}[1]{\frac{\partial}{\partial #1}}

% Common Math Constructs
\newcommand\quotient[2]{\raise1ex\hbox{\ensuremath{#1}}\Big/\lower1ex\hbox{\ensuremath{#2}}} % Quotient of spaces

% Common symbols

% Mathcal
\newcommand{\calA}{\mathcal{A}}
\newcommand{\calB}{\mathcal{B}}
\newcommand{\calC}{\mathcal{C}}
\newcommand{\calD}{\mathcal{D}}
\newcommand{\calE}{\mathcal{E}}
\newcommand{\calF}{\mathcal{F}}
\newcommand{\calG}{\mathcal{G}}
\newcommand{\calH}{\mathcal{H}}
\newcommand{\calI}{\mathcal{I}}
\newcommand{\calJ}{\mathcal{J}}
\newcommand{\calK}{\mathcal{K}}
\newcommand{\calL}{\mathcal{L}}
\newcommand{\calM}{\mathcal{M}}
\newcommand{\calN}{\mathcal{N}}
\newcommand{\calO}{\mathcal{O}}
\newcommand{\calP}{\mathcal{P}}
\newcommand{\calQ}{\mathcal{Q}}
\newcommand{\calR}{\mathcal{R}}
\newcommand{\calS}{\mathcal{S}}
\newcommand{\calT}{\mathcal{T}}
\newcommand{\calU}{\mathcal{U}}
\newcommand{\calV}{\mathcal{V}}
\newcommand{\calW}{\mathcal{W}}
\newcommand{\calX}{\mathcal{X}}
\newcommand{\calY}{\mathcal{Y}}
\newcommand{\calZ}{\mathcal{Z}}


% Delimiters: Based on mathtools

% Declaring paired delimiters like this provides better spacing.
% Normal versions are normal sized delimiters, starred versions are sized delimiters
\usepackage{mathtools}
\usepackage{etoolbox}

% Toggling stared function mapping: https://tex.stackexchange.com/questions/278382/cause-declarepaireddelimiter-to-switch-starred-and-nonstarred-versions 
% Tells DeclarePairedDelimiter to create functions that are automatically sized.
% Starred version is not automatically sized. Both version allow an overriding input.
% Auto sizing:  \abs{x}
% Override:     \abs[\big]{x}
% Not sized:    \abs*{x}
% Override:     \abs*[\big]{x}
%
% Defines two commands, for example: PAIREDabs, abs
% abs checks for a star or options and then calles the right version of PAIREDabs

% If this starts giving too much trouble then I will have to stick with the starred versions.

\makeatletter
\def\DeclarePairedDelimiterStar#1#2#3{%
	\expandafter\DeclarePairedDelimiter\csname PAIRED\string#1\endcsname{#2}{#3}%
	\newcommand#1{%
		\@ifstar{\csname PAIRED\string#1\endcsname}
		{\@ifnextchar[{\csname PAIRED\string#1\endcsname}
			{\csname PAIRED\string#1\endcsname*}%
		}%
	}%
}
\def\DeclarePairedDelimiterStarX#1[#2]#3#4#5{%
	\expandafter\DeclarePairedDelimiterX\csname PAIREDX\string#1\endcsname[#2]{#3}{#4}{#5}%
	\newcommand#1{%
		\@ifstar{\csname PAIREDX\string#1\endcsname}
		{\@ifnextchar[{\csname PAIREDX\string#1\endcsname}
			{\csname PAIREDX\string#1\endcsname*}%
		}%
	}%
}
\def\DeclarePairedDelimiterStarXPP#1[#2]#3#4#5#6#7{%
	\expandafter\DeclarePairedDelimiterXPP\csname PAIREDXPP\string#1\endcsname[#2]{#3}{#4}{#5}{#6}{#7}%
	\newcommand#1{%
		\@ifstar{\csname PAIREDXPP\string#1\endcsname}
		{\@ifnextchar[{\csname PAIREDXPP\string#1\endcsname}
			{\csname PAIREDXPP\string#1\endcsname*}%
		}%
	}%
}
\makeatother

% Not really sure that the micro modification always works.
\DeclarePairedDelimiterStar\brac{(}{)} % Resizable paranthesis
\DeclarePairedDelimiterStar\sbrac{[}{]} % Square brackets
\DeclarePairedDelimiterStar\ceil{\lceil}{\rceil} % Ceiling function
\DeclarePairedDelimiterStar\floor{\lfloor}{\rfloor} % Floor function
\DeclarePairedDelimiterStar\gen{\langle}{\rangle} % Generators
\DeclarePairedDelimiterStar\abs{\lvert}{\rvert} % Absoulte Value
\DeclarePairedDelimiterStar\eval{.}{\rvert} % Integrals
\DeclarePairedDelimiterStarX\norm[1]{\lVert}{\rVert}{\ifblank{#1}{\:\cdot\:}{#1}} % Specialized Norms
\DeclarePairedDelimiterStarX\inprod[2]{\langle}{\rangle}{\ifblank{#1}{\:\cdot\:}{#1},\ifblank{#2}{\:\cdot\:}{#2}} % Inner product

% Copied from mathtools documentation
% We define another local function called given which places this symbol whenever needed.
\providecommand\st{}
\newcommand\stSymbol[1][]{%
	\nonscript\:#1\vert
	\allowbreak
	\nonscript\:
	\mathopen{}
}
% Sets
\DeclarePairedDelimiterStarX\set[1]{\{}{\}}{\renewcommand\st{\stSymbol[\delimsize]}#1}
\DeclarePairedDelimiterStarX\setof[2]{\{}{\}}{\renewcommand\st{\stSymbol[\delimsize]}#1 \st #2}
% Probability 
\DeclarePairedDelimiterStarXPP\pr[1]{\mathbb{P}}{[}{]}{}{\renewcommand\st{\stSymbol[\delimsize]}#1}
\DeclarePairedDelimiterStarXPP\prwith[2]{\mathbb{P}_{#1}}{[}{]}{}{\renewcommand\st{\stSymbol[\delimsize]}#2}
% Expectation
\DeclarePairedDelimiterStarXPP\ex[1]{\mathbb{E}}{[}{]}{}{\renewcommand\st{\stSymbol[\delimsize]}#1}
%\DeclarePairedDelimiterStarXPP\expect[2]{\underset{#1}{\mathbb{E}}}{[}{]}{}{\renewcommand\st{\stSymbol[\delimsize]}#2}
\DeclarePairedDelimiterStarXPP\exwith[2]{\mathbb{E}_{#1}}{[}{]}{}{\renewcommand\st{\stSymbol[\delimsize]}#2}
% Variance
\DeclarePairedDelimiterStarXPP\var[1]{\text{Var}}{[}{]}{}{\renewcommand\st{\stSymbol[\delimsize]}#1}
\DeclarePairedDelimiterStarXPP\varwith[2]{\text{Var}_{#1}}{[}{]}{}{\renewcommand\st{\stSymbol[\delimsize]}#2}

% Operators
\DeclareMathOperator*{\argmin}{argmin}
\DeclareMathOperator*{\argmax}{argmax}
\DeclareMathOperator{\sgn}{sgn} % sign function
\DeclareMathOperator{\sign}{sign} % same thing as above
\DeclareMathOperator{\img}{Im} % image
\DeclareMathOperator{\rank}{rank} % rank
\DeclareMathOperator{\Spec}{Spec} % spectrum
\DeclareMathOperator{\Nil}{Nil} % nilradical
\DeclareMathOperator{\Tr}{Tr} % Trace
\DeclareMathOperator{\pred}{pred} % Predecessor
\DeclareMathOperator{\dom}{dom} % domain
\DeclareMathOperator{\range}{range} % range
\DeclareMathOperator{\Aut}{Aut} % automophisms
\DeclareMathOperator{\Hom}{Hom} % homomorphism
\DeclareMathOperator{\End}{End} % endomorphism
\DeclareMathOperator{\res}{res} % resultant
\DeclareMathOperator{\Int}{Int} % interior
\DeclareMathOperator{\Adj}{Adj} % Adjoint
\DeclareMathOperator{\height}{\mathtt{ht}} % height (mainly for dimension theory of rings)
\DeclareMathOperator{\wt}{\mathtt{wt}} %weight
\DeclareMathOperator{\sing}{Sing} % Singularities
\DeclareMathOperator{\nnz}{nnz} % Non zero entries
\DeclareMathOperator{\poly}{poly} % Polynomial
\DeclareMathOperator{\polylog}{polylog}
\DeclareMathOperator{\conv}{Conv} % Convex Hulls
\DeclareMathOperator{\proj}{proj} % Projections
\DeclareMathOperator{\diam}{diam} % Diameter
\DeclareMathOperator{\supp}{Supp} % Support
\DeclareMathOperator{\spn}{span} % Span
\DeclareMathOperator{\vol}{vol}
\DeclareMathOperator{\opt}{Opt}
\DeclareMathOperator{\lpopt}{LPOpt}
\DeclareMathOperator{\spdopt}{SDPOpt}

\DeclareMathOperator{\ber}{Ber}

\DeclareMathOperator{\dist}{dist} % Distance
\DeclareMathOperator{\KL}{KL}
\DeclareMathOperator{\TV}{TV}

\DeclareMathOperator{\sininv}{\sin^{-1}} % Better Trig Inverses
\DeclareMathOperator{\cosinv}{\cos^{-1}}
\DeclareMathOperator{\taninv}{\tan^{-1}}

% Number systems
\newcommand{\R}{\ensuremath{\mathbb{R}}}
\newcommand{\C}{\ensuremath{\mathbb{C}}}
\newcommand{\N}{\ensuremath{\mathbb{N}}}
\newcommand{\Z}{\ensuremath{\mathbb{Z}}}
\newcommand{\F}{\ensuremath{\mathbb{F}}}
\newcommand{\Q}{\ensuremath{\mathbb{Q}}}

% mathbb shortcuts
\newcommand{\E}{\ensuremath{\mathbb{E}}}

% Matrices
\newcommand{\id}{\ensuremath{\mathrm{I}}}

% Paper specific micros
\newcommand{\nce}{\mathtt{NCE}}

